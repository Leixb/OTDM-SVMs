%! TEX root = **/000-main.tex

\section{Dual}
% Implement the primal and dual quadratic formulation of the support vector machine in AMPL.

To obtain the dual formulation from
the primal from \cref{sec:primal}, we
introduce the Lagrangian function $L$:
\begin{align*}
    L(w,\gamma,s,\lambda,\mu)
    &= \frac{1}{2} w^T w + \nu e^T s -
    % \underbrace{
    \lambda^T \left(Y(Aw + \gamma e) + s - e\right) - \mu^T s
% }_{\text{Duality Gap}}
    % \\
    % &= \frac{1}{2} \sum_{i=1}^n w_i^2 + \nu \sum_{i=1}^m s_i + \sum_{i=1}^m \lambda_i \left(-y_i \left(\phi\left(x_i\right)^T w + \gamma\right) - s_i + 1\right) - \sum_{i=1}^m \mu_i s_i
    % &= \frac{1}{2} \sum_{i=1}^n w_i^2 + \nu \sum_{i=1}^m s_i - \sum_{i=1}^m \lambda_i \left(y_i \left(\sum_{j=1}^n \phi_j\left(x_i\right) w_j + \gamma\right) + s_i - 1\right) - \sum_{i=1}^m \mu_i s_i
\end{align*}
Where $\lambda$ and $\mu$ are the Lagrange multipliers.

We can then obtain the dual formulation by using
Wolfe duality:
\begin{align*}
    \max_{w,\gamma,s,\lambda,\mu} L(w,\gamma,s,\lambda,\mu)
    &= \frac{1}{2} w^T w + \nu e^T s - \lambda^T \left(Y(Aw + \gamma e) + s - e\right) - \mu^T s \\
    \\
    \text{ subj. to: } \\
    \nabla_w L(w,\gamma,s,\lambda,\mu) &= w - \left(\lambda^T YA\right)^T = 0
    \implies w = \left(\lambda^T YA\right)^T \tag{c.1}\label{c1} \\
    \nabla_\gamma L(w,\gamma,s,\lambda,\mu) &= \lambda^T Ye = 0  \tag{c.2}\label{c2}\\
    \nabla_s L(w,\gamma,s,\lambda,\mu) &= \nu e - \lambda + \mu = 0  \tag{c.3}\label{c3}\\
                                       & \lambda \geq 0,\quad \mu \geq 0
\end{align*}

Using that $w = \left(\lambda^T YA\right)^T$, we can
rewrite the objective function as follows:
\begin{align*}
    % \max_{\lambda,\mu}
    & \frac{1}{2}
    \lambda^T YA \left(\lambda^T YA\right)^T
    + \nu e^T s
    + \lambda^T \left(-Y(A\lambda Y^T A^T + \gamma e) - s + e\right)
    - \mu^T s
    \\
    % &= \frac{1}{2}
    % \lambda^T YA \lambda Y^T A^T
    % + \nu e^T s
    % -\lambda^TY(A\lambda Y^T A^T + \gamma e)
    % -\lambda^T s
    % +\lambda^T e
    % - \mu^T s \\
    &= \frac{1}{2}
    \lambda^T YA \lambda Y^T A^T
    + \nu e^T s
    - \lambda^T Y A \lambda Y^T A^T
    - \lambda^T Y \gamma e
    -\lambda^T s
    +\lambda^T e
    - \mu^T s \\
    % &= -\frac{1}{2}
    % \lambda^T Y A \lambda Y^T A^T
    % + \nu e^T s
    % - \lambda^T Y \gamma e
    % -\lambda^T s
    % +\lambda^T e
    % - \mu^T s \\
    &= - \frac{1}{2}
    \lambda^T Y A \lambda Y^T A^T
    + \lambda^T e
    - \cancelto{0 \text{ by }\ref{c2}}{\lambda^T Y \gamma e \vphantom{\frac{1}{2}}}
    + \cancelto{0 \text{ by }\ref{c3}}{(\nu e^T
       - \lambda^T 
   - \mu^T) s}
    \\
    &= - \frac{1}{2}
    \lambda^T Y A \lambda Y^T A^T
    + \lambda^T e
\end{align*}

Finally, using that $\mu \geq 0$ and $\mu = \nu e - \lambda$,
we get that $\lambda \leq \nu e$ and since $e$ is a vector of ones,
$\lambda \leq \nu$. We can thus formulate the dual problem as follows:
\begin{align*}
    \max_{\lambda}\; & \lambda^T e - \frac{1}{2} \lambda^T Y A \lambda Y^T A^T
    \\
    \text{ subj. to: } \\
    \lambda^T Y e & = 0 \\
    0 \leq \lambda & \leq \nu
\end{align*}


\pagebreak
\subsection{AMPL Model}

As we mentioned in \cref{sec:primal}, we need to use
the scalar form of the dual formulation to be able to
implement it in AMPL.

The scalar form of the dual formulation is:
\begin{align*}
    \max_{\lambda}\; & \sum_{i=1}^m \lambda_i - 
    \mathlarger{\frac{1}{2}
    \sum_{i=1}^m}
    \sum_{j=1}^m \lambda_i \lambda_j y_i y_j K_{ij}
    \\
    \text{ subj. to: } \\
    \sum_{i=1}^m \lambda_i y_i & = 0 \\
    0 \leq \lambda_i & \leq \nu \qquad \forall i \in \{1,\ldots,m\}
\end{align*}
where $K = AA^T$.

We can model this in AMPL as follows:

\begin{listing}[H]
    \caption{AMPL Dual SVM model (\texttt{dual.mod})}
    \inputminted{ampl}{../ampl/dual.mod}
\end{listing}
