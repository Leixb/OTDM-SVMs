%! TEX root = **/000-main.tex
% vim: spell spelllang=en:

%%%%%%%%%%%%%%%%%%%%%%%%%%%%%%%%%%%%%%%%%%%%%%%%%%%%%%%%%%%%%%%%%%%%%%%%%%%%%%%%
% PREAMBLE
%%%%%%%%%%%%%%%%%%%%%%%%%%%%%%%%%%%%%%%%%%%%%%%%%%%%%%%%%%%%%%%%%%%%%%%%%%%%%%%%
\input{001-preamble}

%%%%%%%%%%%%%%%%%%%%%%%%%%%%%%%%%%%%%%%%%%%%%%%%%%%%%%%%%%%%%%%%%%%%%%%%%%%%%%%%
% EXTRA PACKAGES / CONFIG
%%%%%%%%%%%%%%%%%%%%%%%%%%%%%%%%%%%%%%%%%%%%%%%%%%%%%%%%%%%%%%%%%%%%%%%%%%%%%%%%

%%%%%%%%%%%%%%%%%%%%%%%%%%%%%%%%%%%%%%%%%%%%%%%%%%%%%%%%%%%%%%%%%%%%%%%%%%%%%%%%
% METADATA
%%%%%%%%%%%%%%%%%%%%%%%%%%%%%%%%%%%%%%%%%%%%%%%%%%%%%%%%%%%%%%%%%%%%%%%%%%%%%%%%

% remove when using \maketitle:
\renewcommand\and{\\[\baselineskip]}

\title{Support Vector Machines}
\author{Aleix Boné}
\date{Fall 2022}

\begin{document}
%%%%%%%%%%%%%%%%%%%%%%%%%%%%%%%%%%%%%%%%%%%%%%%%%%%%%%%%%%%%%%%%%%%%%%%%%%%%%%%%
% TITLE
%%%%%%%%%%%%%%%%%%%%%%%%%%%%%%%%%%%%%%%%%%%%%%%%%%%%%%%%%%%%%%%%%%%%%%%%%%%%%%%%

% Default title or use titlepage.tex

%\maketitle
\pagestyle{empty}

\makeatletter
\begin{tikzpicture}[
		remember picture,
		overlay,
		important line/.style={thick,BrickRed!15,thick},
		dashed line/.style={dashed,BrickRed!15,thick},
		leftNode/.style={circle,minimum width=.5ex, fill=BrickRed!15,draw},
		rightNode/.style={rectangle,minimum width=.5ex, fill=BrickRed!15,thick,draw},
	]
	%%%%%%%%%%%%%%%%%%%% Background %%%%%%%%%%%%%%%%%%%%%%%%
    \begin{scope}[blend mode=hard light]
	\fill[BrickRed] (current page.south west) rectangle (current page.north east);

    \node[anchor=north,inner sep=0pt] at ($(current page.north)-(0,1)$) {
        \includegraphics[width=0.9\textwidth,
        ]{logo-white}
    };
    \end{scope}

	\pgfmathsetseed{1234}
	\begin{axis}[
			at={(current page.south west)},
			width=\paperwidth,
			height=\paperheight,
            xmin=-100,xmax=50,
            clip=false,
            ymin=-50,ymax=200,
            ticks=none,
            axis lines=none,
		]
		\pgfmathsetmacro{\sep}{20};
		\pgfmathsetmacro{\xorig}{120};
		\pgfmathsetmacro{\rot}{-60};

		\begin{scope}[rotate around={\rot:(0,0)}]
			\draw[dashed line] (-300, -\sep) -- (300, -\sep);
			\draw[dashed line] (-300, \sep) -- (300, \sep);
			\draw[important line] (-300, 0) -- (300, 0);

            \node[leftNode,label={[BrickRed!3]:$x_1$},name=x1] at (-20, \sep) {};
			\node[rightNode,label={[BrickRed!3]$x_2$},name=x2] at (-22, -\sep) {};

			\node[leftNode,label={[BrickRed!3]$x_3$},name=x3] at (-10, -\sep*3/2) {};
			\node[leftNode,label={[BrickRed!3]$x_5$},name=x5] at (-27, -\sep/3) {};

			\node[rightNode,label={[BrickRed!3]$x_4$},name=x4] at (10, -\sep*2/3) {};

			\coordinate (origin) at (0, 0);
			\coordinate (left) at (0, \sep);
			\coordinate (right) at (0, -\sep);

			\coordinate (x3l) at (x3 |- left);
			\coordinate (x4r) at (x4 |- right);
			\coordinate (x5l) at (x5 |- left);

			\begin{scope}[color=BrickRed!10]
				\draw (x3) -- (x3l) {};
				\draw (x4) -- (x4r) {};
				\draw (x5) -- (x5l) {};
				\node[anchor=north] at ($(x3)!0.5!(x3l)$) {$\xi_3$};
				\node[anchor=north] at ($(x4)!0.5!(x4r)$) {$\xi_4$};
				\node[anchor=north] at ($(x5)!0.5!(x5l)$) {$\xi_5$};
			\end{scope}

			\draw[<->,BrickRed!5,line width=2pt] (-44,\sep) -- (-44,-\sep);
			\node[BrickRed!9,anchor=south,rotate=\rot+90] at (-44.25,0) {margin\;\; = $\frac{2}{\lVert \omega \rVert}$};

			\pgfplotsinvokeforeach{0.00,0.1,...,1.00}{
				\node [leftNode] at (rand*40-20,\sep+rnd*40) {};
				\node [rightNode] at (rand*40,-\sep-rnd*40) {};
			}

            \coordinate (p1) at (-70, \sep);
            \coordinate (p0) at (-70, 0);
            \coordinate (p-1) at (-70, -\sep);

		\end{scope}
		\node[BrickRed!5,anchor=south east,rotate=\rot] at (p1) {$\omega^T x + b = 1$};
		\node[BrickRed!5,anchor=south east,rotate=\rot] at (p0) {$\pi:\,\omega^T x + b = 0$};
		\node[BrickRed!5,anchor=south east,rotate=\rot] at (p-1) {$\omega^T x + b = -1$};
	\end{axis}

	% \foreach \i in {2.5,...,22}
	% {
	%     \node[rounded corners,BrickRed!60,draw,regular polygon,regular polygon sides=7, minimum size=\i cm,ultra thick] at ($(current page.west)+(2.5,-5)$) {} ;
	% }

	% %%%%%%%%%%%%%%%%%%%% Background Polygon %%%%%%%%%%%%%%%%%%%%
	% \foreach \i in {0.5,...,22}
	% {
	% \node[rounded corners,BrickRed!60,draw,regular polygon,regular polygon sides=7, minimum size=\i cm,ultra thick] at ($(current page.north west)+(2.5,0)$) {} ;
	% }

	% \foreach \i in {0.5,...,22}
	% {
	% \node[rounded corners,BrickRed!90,draw,regular polygon,regular polygon sides=7, minimum size=\i cm,ultra thick] at ($(current page.north east)+(0,-9.5)$) {} ;
	% }


	% \foreach \i in {21,...,6}
	% {
	% \node[BrickRed!85,rounded corners,draw,regular polygon,regular polygon sides=7, minimum size=\i cm,ultra thick] at ($(current page.south east)+(-0.2,-0.45)$) {} ;
	% }


	%%%%%%%%%%%%%%%%%%%% Title of the Report %%%%%%%%%%%%%%%%%%%%
	\node[left,BrickRed!5,minimum width=0.725*\paperwidth,minimum height=3cm, rounded corners,align=center] at ($(current page.north east)+(0,-6.0)$)
	{
    {\fontsize{25}{30} \selectfont \bfseries Support}
	};
	\node[left,BrickRed!5,minimum width=0.725*\paperwidth,minimum height=3cm, rounded corners,align=center] at ($(current page.north east)+(0,-7.5)$)
	{
    {\fontsize{25}{30} \selectfont \bfseries Vector Machines}
	};

	%%%%%%%%%%%%%%%%%%%% Subtitle %%%%%%%%%%%%%%%%%%%%
	\node[left,BrickRed!10,minimum width=0.725*\paperwidth,minimum height=2cm, rounded corners] at ($(current page.north east)+(0,-9)$)
	{
		{\huge \textit{OTDM}}
	};

	%%%%%%%%%%%%%%%%%%%% Author Name %%%%%%%%%%%%%%%%%%%%
	\node[left,BrickRed!5,minimum width=0.725*\paperwidth,minimum height=2cm, rounded corners] at ($(current page.north east)+(0,-11)$)
	{
		{\Large \textsc{\@author}}
	};

	%%%%%%%%%%%%%%%%%%%% Year %%%%%%%%%%%%%%%%%%%%
\node[rounded corners,fill=BrickRed!70,text =BrickRed!5,regular polygon,regular polygon sides=6, minimum size=2.5 cm,inner sep=0,ultra thick,align=center] at ($(current page.west)+(2.5,-5)$) {\LARGE \bfseries 2022};


\end{tikzpicture}
\makeatother

\include{005-titlepage}

%%%%%%%%%%%%%%%%%%%%%%%%%%%%%%%%%%%%%%%%%%%%%%%%%%%%%%%%%%%%%%%%%%%%%%%%%%%%%%%%
% TOC & lists
%%%%%%%%%%%%%%%%%%%%%%%%%%%%%%%%%%%%%%%%%%%%%%%%%%%%%%%%%%%%%%%%%%%%%%%%%%%%%%%%

\pagenumbering{Roman}

%\setcounter{tocdepth}{2}
\tableofcontents \pagebreak

\pagenumbering{arabic}

%%%%%%%%%%%%%%%%%%%%%%%%%%%%%%%%%%%%%%%%%%%%%%%%%%%%%%%%%%%%%%%%%%%%%%%%%%%%%%%%
% SECTIONS
%%%%%%%%%%%%%%%%%%%%%%%%%%%%%%%%%%%%%%%%%%%%%%%%%%%%%%%%%%%%%%%%%%%%%%%%%%%%%%%%

% Paragraph spacing (placed after ToC)
\setlength{\parskip}{1em plus 0.5em minus 0.2em}
%\setlength{\parindent}{0pt}

\setlength{\headheight}{14.5pt}
\pagestyle{fancy}


% Implement the primal and dual quadratic formulation of the support vector machine in AMPL.

% Apply to a dataset obtained with the accompanying generator.
% Validate the SVM with data different from that of the training set.

% Optionally (but highly recommended for a good mark) you can apply it to other datasets.

% Compute the separation hyperplane from the dual model and check that it coincides with that of the primal model.

% The report must include all the previous elements (AMPL code, results obtained, analysis of results, etc).


%! TEX root = **/000-main.tex
\chapter{Primal Formulation}


%! TEX root = **/000-main.tex
\chapter{Dual Formulation}

%! TEX root = **/000-main.tex
\chapter{Validation}

\section{Accuracy metric}

To validate our implementation, we
need to apply it to a dataset and measure
the performance of the SVM. To that end,
we use the \texttt{accuracy} metric.

\subsection{Accuracy on the Primal formulation}

In order to compute the accuracy, we
need to know the prediction of the SVM, we
need to compute the sign of the decision
function, which is given by:
\begin{equation*}
	\label{eq:categories}
	y_i = \begin{cases}
        \hphantom{-}1 & \text{if } \sum_{j=1}^n A_{ij} w_j + \gamma \geq 0 \\
		-1 & \text{otherwise}
	\end{cases}
\end{equation*}

Using this, we can compute the accuracy in AMPL:
\begin{listing}[H]
	\caption{AMPL Accuracy (\texttt{accuracy.run})}
    \inputminted{ampl}{../ampl/accuracy.run}
\end{listing}

Note, that this script uses \texttt{A} and \texttt{y}, which
correspond to the training set. However, we can override
the values by using \mintinline[firstline=1]{ampl}{reset data A, y}
and setting the values from the test set. This is done so
that we can use the same script for both the training and
test set metrics.

\subsection{Accuracy on the Dual formulation}%
\label{sec:accuracy-dual}

The accuracy on the dual formulation can be computed by
checking the values of the $\lambda$ variables. We know that
misclassified points are equal to $\nu$.
\begin{listing}[H]
	\caption{AMPL Accuracy for the Dual (\texttt{accuracy\_dual.run})}
    \label{lst:accuracy-dual}
    \inputminted{ampl}{../ampl/accuracy_dual.run}
\end{listing}

\Cref{lst:accuracy-dual} shows the script for computing the accuracy
on AMPL. We use a tolerance of $10^{-8}$ when checking for
$\lambda \neq \nu$ to mitigate numerical errors.

However, with this approach, we are not able to compute the
accuracy on the test set since we do not have access to the
$\lambda$ variables. In order to compute the accuracy on the
test set, we need to compute the separation hyperplane ($w$ and $\gamma$)
of the dual formulation.

\subsection{Separation Hyperplane from the dual}%
% Compute the separation hyperplane from the dual model and check that it coincides with that of the primal model.

We can obtain $w$ and $\gamma$ from the dual formulation
by using the following equations:
\begin{alignat*}{2}
    w_j &= \sum_{i=1}^m \lambda_i y_i A_{ij},&\quad j &\in \{1,\ldots,n\} \\
    % \gamma &= \frac{1}{y_i} - \sum_{j=1}^{n} w_j A_{ij},&\quad j &\in \{1,\ldots,n\},\;i \in SV
    \gamma &= \frac{1}{n}\left(y_i - \sum_{j=1}^{n} w_j A_{ij}\right),&\quad j &\in \{1,\ldots,n\},\;i \in SV
\end{alignat*}
where $SV$ is the set of \emph{support vectors} (i.e. $s_i = 0,\, \lambda_i > 0$).

This can be implemented in AMPL as shown in \cref{lst:dual_plane}. Note
that we use a tolerance of $10^{-2}$ when doing comparisons to mitigate
numerical errors. The choice of $10^{-2}$ is based on the value that
we found to give the best results (i.e. equal to the primal formulation)
in our experiments as we will see in \cref{sec:nu-accuracy}.
\begin{listing}[H]
	\caption{AMPL normal vector and intercept for dual (\texttt{dual\_plane.run})}
    \label{lst:dual_plane}
    \inputminted{ampl}{../ampl/dual_plane.run}
\end{listing}

Once we have the normal vector and the intercept, we can
compute the accuracy using the same procedure as in
the primal formulation.

\subsubsection{Equivalence of the primal and dual formulations}%

Using the computation of the normal vector and the intercept
from the dual formulation shown in \cref{sec:accuracy-dual},
we can check that the primal and dual formulations are
equivalent. That is, the values of $w$ and $\gamma$ should be
the same (within tolerance) in both cases.


\pagebreak
\section{Results}

Now, we can put it all together and run the model on the
dataset and compute the accuracy on the training and test
for both the primal and dual formulations.

To run the model, we use the following script:
\begin{listing}[H]
    \caption{AMPL script to run the primal model (\texttt{primal.run})}
    \inputminted{ampl}{../ampl/primal.run}
\end{listing}

The script for the dual formulation is identical, except
that we use \mintinline[firstline=1]{ampl}{model ./dual.mod} instead of
\mintinline[firstline=1]{ampl}{model ./primal.mod} and we add 
\mintinline[firstline=1]{ampl}{include ./dual_plane.run} after the
solver call to compute the normal vector and the intercept.
% \inputminted{ampl}{../ampl/dual.run}


\pagebreak
\subsection{Generated dataset}
% Apply to a dataset obtained with the accompanying generator.
% Validate the SVM with data different from that of the training set.

Using the dataset provided with the generator \texttt{gensvmdat},
we created a training set with $m=1\,000$ samples and a test set
with $m=500$ samples\footnote{Seeds: 458937 (training), 789453 (test)}.


\begin{listing}[H]
\inputminted[firstline=1,bgcolor=lightcodeBg]{text}{../outputs/primal.out}
\caption{Primal result (Output of: \texttt{ampl primal.run})}
\label{lst:res_primal}
\end{listing}

\begin{listing}[H]
\inputminted[firstline=1,bgcolor=lightcodeBg]{text}{../outputs/dual.out}
\caption{Dual result (Output of: \texttt{ampl dual.run})}
\label{lst:res_dual}
\end{listing}

As we can see from \cref{lst:res_primal,lst:res_dual},
the primal and dual formulations have
reached the same separation hyperplane.

\subsection{Ionosphere dataset}
% Optionally (but highly recommended for a good mark) you can apply it to other datasets.
We also applied the model to the well known \emph{Ionosphere} dataset%
\cite{noauthor_uci_nodate}. This dataset contains $m=351$ samples
and $n=34$ features. 

The dataset provides radar data and the goal is to classify
the data as either \emph{good} or \emph{bad}. ``Good'' refers to
data that shows evidence of an object, while ``bad'' refers to
data that does not show evidence of an object.

For the purpose of this project, we set the target value of good
as $y_i = 1$ and the target value of bad as $y_i = -1$.

The target variable, is not evenly distributed, with 225 samples
being good and 126 samples being bad. Therefore, we used a
\emph{stratified} 5:1 split to create the training and test sets.

\begin{listing}[H]
\inputminted[firstline=1,bgcolor=lightcodeBg]{text}{../outputs/primal_io.out}
\caption{Primal result (Output of: \texttt{ampl primal\_io.run})}
\label{lst:res_primal_io}
\end{listing}

\begin{listing}[H]
\inputminted[firstline=1,bgcolor=lightcodeBg]{text}{../outputs/dual_io.out}
\caption{Dual result (Output of: \texttt{ampl dual\_io.run})}
\label{lst:res_dual_io}
\end{listing}

\Cref{lst:res_primal_io,lst:res_dual_io} show the results of the
primal and dual formulations on the Ionosphere dataset. As before,
the primal and dual formulations have reached the same separation
hyperplane. There are slight differences in the values, but they
below $10^{-5}$ and do not affect the classification results.

%! TEX root = **/000-main.tex
\chapter{Effect of the $\nu$ parameter}

Both the primal and dual formulation of the support vector machine
have a parameter $\nu$ that controls the trade-off.

Since we did not know a priori what vale of $\nu$ to choose,
we decided to write a script to try different values of $\nu$ and
compare the accuracy and runtime of the models.

\section{Accuracy}

\begin{figure}[H]
    \begin{tikzpicture}
        \begin{axis}[
            table/col sep=comma,
            xmode=log,
            xlabel=$\nu$,
            ylabel=Accuracy (test),
            ymax=1,
            axis y discontinuity=parallel,
            width=0.7\textwidth,
            height=0.4\textwidth,
            ]
            \addplot+ table[x index=0, y index=1] {../benchmarks/primal.csv};
            \addlegendentry{Primal};
            \addplot+ table[x index=0, y index=1] {../benchmarks/dual.csv};
            \addlegendentry{Dual};
        \end{axis}
    \end{tikzpicture}
    \caption{Accuracy of the primal and dual formulations}
\end{figure}

For values of $\nu$ between $10^{-2}$ and $10^{4}$, the primal
and dual models obtain the same hyperplane and therefore the same accuracy.

However, for $\nu > 10^{4}$, the dual model starts to get different
results than the primal model and its accuracy decreases. This should
not happen since the duality gap is theoretically zero. Upon further
analysis, we found that the cause was the calculation of
the gamma parameter in the dual model.

The formula we used was
sensible to small numerical errors: the comparison
to determine if a point was a support vector was done using
$\lambda_i > 0$ and $\lambda_i < \nu$, which is problematic
due to floating point errors. We fixed this by comparisons
of the form: $\lambda_i > \varepsilon$ and $\nu - \lambda_i > \varepsilon$,
where $\varepsilon$ is a small number. We found that $\varepsilon = 10^{-2}$
gave good results even with large values of $\nu$.
In line 11 of~\cref{lst:dual_plane} we can see the final formula in AMPL.

With this fix, the dual model behaves as expected (equally to the primal model)
for all values of $\nu$ between $10^{-2}$ and $10^{7}$.

\section{Runtime}

We used \texttt{hyperfine}%
\cite{peter_hyperfine_2022}
to measure the runtime of the primal and dual models. Below
we show the results obtained for each model.


\begin{listing}[H]
\begin{minted}[firstline=1,bgcolor=blue!2]{shell-session}
$ hyperfine "ampl primal.run"
Benchmark 1: ampl primal.run
  Time (mean ± σ):      33.9 ms ±   0.6 ms    [User: 33.8 ms, System: 15.2 ms]
  Range (min … max):    33.1 ms …  36.9 ms    77 runs
$ hyperfine "ampl dual.run"
Benchmark 1: ampl dual.run
  Time (mean ± σ):      2.519 s ±  0.024 s    [User: 2.161 s, System: 0.390 s]
  Range (min … max):    2.498 s …  2.576 s    10 runs
\end{minted}
\caption{Runtime of the primal and dual models for \emph{Ionosphere} dataset}
\end{listing}

\begin{listing}[H]
\begin{minted}[firstline=1,bgcolor=blue!2]{shell-session}
$ hyperfine "ampl primal_io.run"
Benchmark 1: ampl primal_io.run
  Time (mean ± σ):      60.2 ms ±   2.2 ms    [User: 146.2 ms, System: 98.5 ms]
  Range (min … max):    57.3 ms …  67.3 ms    46 runs
$ hyperfine "ampl dual_io.run"
Benchmark 1: ampl dual_io.run
  Time (mean ± σ):     406.8 ms ±  29.0 ms    [User: 372.6 ms, System: 42.4 ms]
  Range (min … max):   378.5 ms … 482.4 ms    10 runs
\end{minted}
\caption{Runtime of the primal and dual models for \emph{Ionosphere} dataset}
\end{listing}

We found that for our dataset, the primal model is
significantly faster than the dual model. This is expected
since our dataset is small and has low dimensionality.
With larger datasets, the dual model should be faster.

The parameter $\nu$ did not seem to have any noticeable effect on the runtime
in our limited test.


\chapter*{Conclusions}
\addcontentsline{toc}{chapter}{Conclusions}
We successfully implemented the primal and dual quadratic formulations of the support vector machine in AMPL.
We applied the model to a training dataset obtained with the accompanying generator and validated
it with a test set.

We also applied the model to the \emph{Ionosphere} dataset and obtained good results.

The computed the separation hyperplane from the dual model coincided with that of the primal model.
However, we had to modify the initial computation of the separation hyperplane to account
for numerical errors.

For the datasets we used, the primal model was significantly faster than the dual model. However,
we did not perform any extensive benchmarking and for larger datasets the dual model should be faster.

\nocite{*}

\printbibliography

\end{document}
