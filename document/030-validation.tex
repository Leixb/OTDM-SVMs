%! TEX root = **/000-main.tex
\chapter{Validation}

\section{Accuracy metric}

To validate our implementation, we
need to apply it to a dataset and measure
the performance of the SVM. To that end,
we use the \texttt{accuracy} metric.

\subsection{Accuracy on the Primal formulation}

In order to compute the accuracy, we
need to know the prediction of the SVM, we
need to compute the sign of the decision
function, which is given by:
\begin{equation*}
	\label{eq:categories}
	y_i = \begin{cases}
        \hphantom{-}1 & \text{if } \sum_{j=1}^n A_{ij} w_j + \gamma \geq 0 \\
		-1 & \text{otherwise}
	\end{cases}
\end{equation*}

Using this, we can compute the accuracy in AMPL:
\begin{listing}[H]
	\caption{AMPL Accuracy (\texttt{accuracy.run})}
    \inputminted{ampl}{../ampl/accuracy.run}
\end{listing}

Note, that this script uses \texttt{A} and \texttt{y}, which
correspond to the training set. However, we can override
the values by using \mintinline[firstline=1]{ampl}{reset data A, y}
and setting the values from the test set. This is done so
that we can use the same script for both the training and
test set metrics.

\subsection{Accuracy on the Dual formulation}%
\label{sec:accuracy-dual}

We can obtain $w$ and $\gamma$ from the dual formulation
by using the following equations:
\begin{alignat*}{2}
    w_j &= \sum_{i=1}^m \lambda_i y_i A_{ij},&\quad j &\in \{1,\ldots,n\} \\
    % \gamma &= \frac{1}{y_i} - \sum_{j=1}^{n} w_j A_{ij},&\quad j &\in \{1,\ldots,n\},\;i \in SV
    \gamma &= \frac{1}{n}\left(y_i - \sum_{j=1}^{n} w_j A_{ij}\right),&\quad j &\in \{1,\ldots,n\},\;i \in SV
\end{alignat*}
where $SV$ is the set of \emph{support vectors} (i.e. $s_i = 0,\, \lambda_i > 0$).

\begin{listing}[H]
	\caption{AMPL normal vector and intercept for dual (\texttt{dual\_plane.run})}
    \label{lst:dual_plane}
    \inputminted{ampl}{../ampl/dual_plane.run}
\end{listing}

Once we have the normal vector and the intercept, we can
compute the accuracy as in the primal case.

\subsection{Equivalence of the primal and dual formulations}
% Compute the separation hyperplane from the dual model and check that it coincides with that of the primal model.
Using the computation of the normal vector and the intercept
from the dual formulation shown in \cref{sec:accuracy-dual},
we can check that the primal and dual formulations are
equivalent. That is, the values of $w$ and $\gamma$ should be
the same in both cases.


\pagebreak
\section{Results}

Now, we can put it all together and run the model on the
dataset and compute the accuracy on the training and test
for both the primal and dual formulations.

To run the model, we use the following script:
\begin{listing}[H]
    \caption{AMPL script to run the primal model (\texttt{primal.run})}
    \inputminted{ampl}{../ampl/primal.run}
\end{listing}

The script for the dual formulation is identical, except
that we use \mintinline[firstline=1]{ampl}{model ./dual.mod} instead of
\mintinline[firstline=1]{ampl}{model ./primal.mod} and we add 
\mintinline[firstline=1]{ampl}{include ./dual_plane.run} after the
solver call to compute the normal vector and the intercept.
% \inputminted{ampl}{../ampl/dual.run}


\pagebreak
\subsection{Artificial dataset}
% Apply to a dataset obtained with the accompanying generator.
% Validate the SVM with data different from that of the training set.

Using the dataset provided with the generator \texttt{gensvmdat},
we created a training set with $m=1\,000$ samples and a test set
with $m=500$ samples\footnote{Seeds: 458937 (training), 789453 (test)}.


\begin{listing}[H]
\inputminted[firstline=1,bgcolor=lightcodeBg]{text}{../outputs/primal.out}
\caption{Primal result (Output of: \texttt{ampl primal.run})}
\label{lst:res_primal}
\end{listing}

\begin{listing}[H]
\inputminted[firstline=1,bgcolor=lightcodeBg]{text}{../outputs/dual.out}
\caption{Dual result (Output of: \texttt{ampl dual.run})}
\label{lst:res_dual}
\end{listing}

As we can see from \cref{lst:res_primal,lst:res_dual},
the primal and dual formulations have
reached the same separation hyperplane.

\subsection{Ionosphere dataset}
% Optionally (but highly recommended for a good mark) you can apply it to other datasets.
We also applied the model to the well known \emph{Ionosphere} dataset%
\cite{noauthor_uci_nodate}. This dataset contains $m=351$ samples
and $n=34$ features. 

The dataset provides radar data and the goal is to classify
the data as either \emph{good} or \emph{bad}. ``Good'' refers to
data that shows evidence of an object, while ``bad'' refers to
data that does not show evidence of an object.

For the purpose of this project, we set the target value of good
as $y_i = 1$ and the target value of bad as $y_i = -1$.

The target variable, is not evenly distributed, with 225 samples
being good and 126 samples being bad. Therefore, we used a
\emph{stratified} 5:1 split to create the training and test sets.

\begin{listing}[H]
\inputminted[firstline=1,bgcolor=lightcodeBg]{text}{../outputs/primal_io.out}
\caption{Primal result (Output of: \texttt{ampl primal\_io.run})}
\label{lst:res_primal_io}
\end{listing}

\begin{listing}[H]
\inputminted[firstline=1,bgcolor=lightcodeBg]{text}{../outputs/dual_io.out}
\caption{Dual result (Output of: \texttt{ampl dual\_io.run})}
\label{lst:res_dual_io}
\end{listing}

\Cref{lst:res_primal_io,lst:res_dual_io} show the results of the
primal and dual formulations on the Ionosphere dataset. As before,
the primal and dual formulations have reached the same separation
hyperplane. There are slight differences in the values, but they
below $10^{-5}$ and do not affect the classification results.
