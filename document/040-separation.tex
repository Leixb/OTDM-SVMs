%! TEX root = **/000-main.tex
\chapter{Effect of the $\nu$ parameter}

Both the primal and dual formulation of the support vector machine
have a parameter $\nu$ that controls the trade-off.

Since we did not know a priori what vale of $\nu$ to choose,
we decided to write a script to try different values of $\nu$ and
compare the accuracy and runtime of the models.

\section{Accuracy}

\begin{figure}[H]
    \begin{tikzpicture}
        \begin{axis}[
            table/col sep=comma,
            xmode=log,
            xlabel=$\nu$,
            ylabel=Accuracy (test),
            ymax=1,
            axis y discontinuity=parallel,
            width=0.7\textwidth,
            height=0.4\textwidth,
            ]
            \addplot+ table[x index=0, y index=1] {../benchmarks/primal.csv};
            \addlegendentry{Primal};
            \addplot+ table[x index=0, y index=1] {../benchmarks/dual.csv};
            \addlegendentry{Dual};
        \end{axis}
    \end{tikzpicture}
    \caption{Accuracy of the primal and dual formulations}
\end{figure}

For values of $\nu$ between $10^{-2}$ and $10^{4}$, the primal
and dual models obtain the same hyperplane and therefore the same accuracy.

However, for $\nu > 10^{4}$, the dual model starts to get different
results than the primal model and its accuracy decreases. This should
not happen since the duality gap is theoretically zero. Upon further
analysis, we found that the cause was the calculation of
the gamma parameter in the dual model.

The formula we used was
sensible to small numerical errors: the comparison
to determine if a point was a support vector was done using
$\lambda_i > 0$ and $\lambda_i < \nu$, which is problematic
due to floating point errors. We fixed this by comparisons
of the form: $\lambda_i > \varepsilon$ and $\nu - \lambda_i > \varepsilon$,
where $\varepsilon$ is a small number. We found that $\varepsilon = 10^{-2}$
gave good results even with large values of $\nu$.
In line 11 of~\cref{lst:dual_plane} we can see the final formula in AMPL.

With this fix, the dual model behaves as expected (equally to the primal model)
for all values of $\nu$ between $10^{-2}$ and $10^{7}$.

\section{Runtime}

We used \texttt{hyperfine}%
\cite{peter_hyperfine_2022}
to measure the runtime of the primal and dual models. Below
we show the results obtained for each model.


\begin{listing}[H]
\begin{minted}[firstline=1,bgcolor=blue!2]{shell-session}
$ hyperfine "ampl primal.run"
Benchmark 1: ampl primal.run
  Time (mean ± σ):      33.9 ms ±   0.6 ms    [User: 33.8 ms, System: 15.2 ms]
  Range (min … max):    33.1 ms …  36.9 ms    77 runs
$ hyperfine "ampl dual.run"
Benchmark 1: ampl dual.run
  Time (mean ± σ):      2.519 s ±  0.024 s    [User: 2.161 s, System: 0.390 s]
  Range (min … max):    2.498 s …  2.576 s    10 runs
\end{minted}
\caption{Runtime of the primal and dual models for \emph{Ionosphere} dataset}
\end{listing}

\begin{listing}[H]
\begin{minted}[firstline=1,bgcolor=blue!2]{shell-session}
$ hyperfine "ampl primal_io.run"
Benchmark 1: ampl primal_io.run
  Time (mean ± σ):      60.2 ms ±   2.2 ms    [User: 146.2 ms, System: 98.5 ms]
  Range (min … max):    57.3 ms …  67.3 ms    46 runs
$ hyperfine "ampl dual_io.run"
Benchmark 1: ampl dual_io.run
  Time (mean ± σ):     406.8 ms ±  29.0 ms    [User: 372.6 ms, System: 42.4 ms]
  Range (min … max):   378.5 ms … 482.4 ms    10 runs
\end{minted}
\caption{Runtime of the primal and dual models for \emph{Ionosphere} dataset}
\end{listing}

We found that for our dataset, the primal model is
significantly faster than the dual model. This is expected
since our dataset is small and has low dimensionality.
With larger datasets, the dual model should be faster.

The parameter $\nu$ did not seem to have any noticeable effect on the runtime
in our limited test.
